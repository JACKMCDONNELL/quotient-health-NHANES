% Options for packages loaded elsewhere
\PassOptionsToPackage{unicode}{hyperref}
\PassOptionsToPackage{hyphens}{url}
\documentclass[
]{article}
\usepackage{xcolor}
\usepackage{amsmath,amssymb}
\setcounter{secnumdepth}{-\maxdimen} % remove section numbering
\usepackage{iftex}
\ifPDFTeX
  \usepackage[T1]{fontenc}
  \usepackage[utf8]{inputenc}
  \usepackage{textcomp} % provide euro and other symbols
\else % if luatex or xetex
  \usepackage{unicode-math} % this also loads fontspec
  \defaultfontfeatures{Scale=MatchLowercase}
  \defaultfontfeatures[\rmfamily]{Ligatures=TeX,Scale=1}
\fi
\usepackage{lmodern}
\ifPDFTeX\else
  % xetex/luatex font selection
\fi
% Use upquote if available, for straight quotes in verbatim environments
\IfFileExists{upquote.sty}{\usepackage{upquote}}{}
\IfFileExists{microtype.sty}{% use microtype if available
  \usepackage[]{microtype}
  \UseMicrotypeSet[protrusion]{basicmath} % disable protrusion for tt fonts
}{}
\makeatletter
\@ifundefined{KOMAClassName}{% if non-KOMA class
  \IfFileExists{parskip.sty}{%
    \usepackage{parskip}
  }{% else
    \setlength{\parindent}{0pt}
    \setlength{\parskip}{6pt plus 2pt minus 1pt}}
}{% if KOMA class
  \KOMAoptions{parskip=half}}
\makeatother
\usepackage{longtable,booktabs,array}
\newcounter{none} % for unnumbered tables
\usepackage{calc} % for calculating minipage widths
% Correct order of tables after \paragraph or \subparagraph
\usepackage{etoolbox}
\makeatletter
\patchcmd\longtable{\par}{\if@noskipsec\mbox{}\fi\par}{}{}
\makeatother
% Allow footnotes in longtable head/foot
\IfFileExists{footnotehyper.sty}{\usepackage{footnotehyper}}{\usepackage{footnote}}
\makesavenoteenv{longtable}
\usepackage{graphicx}
\makeatletter
\newsavebox\pandoc@box
\newcommand*\pandocbounded[1]{% scales image to fit in text height/width
  \sbox\pandoc@box{#1}%
  \Gscale@div\@tempa{\textheight}{\dimexpr\ht\pandoc@box+\dp\pandoc@box\relax}%
  \Gscale@div\@tempb{\linewidth}{\wd\pandoc@box}%
  \ifdim\@tempb\p@<\@tempa\p@\let\@tempa\@tempb\fi% select the smaller of both
  \ifdim\@tempa\p@<\p@\scalebox{\@tempa}{\usebox\pandoc@box}%
  \else\usebox{\pandoc@box}%
  \fi%
}
% Set default figure placement to htbp
\def\fps@figure{htbp}
\makeatother
\setlength{\emergencystretch}{3em} % prevent overfull lines
\providecommand{\tightlist}{%
  \setlength{\itemsep}{0pt}\setlength{\parskip}{0pt}}
\usepackage{bookmark}
\IfFileExists{xurl.sty}{\usepackage{xurl}}{} % add URL line breaks if available
\urlstyle{same}
\hypersetup{
  hidelinks,
  pdfcreator={LaTeX via pandoc}}

\author{}
\date{}

\begin{document}

\subsection{\texorpdfstring{\textbf{Longevity Score --- NHANES
Validation Summary (Draft for Internal
Review)}}{Longevity Score --- NHANES Validation Summary (Draft for Internal Review)}}\label{longevity-score-nhanes-validation-summary-draft-for-internal-review}

\begin{center}\rule{0.5\linewidth}{0.5pt}\end{center}

\subsection{\texorpdfstring{\textbf{1.
Overview}}{1. Overview}}\label{overview}

This document summarizes the first-pass validation of the
\textbf{Longevity Score} using NHANES 2015--2016. My primary goal was to
determine how well the score tracks with physiologic vulnerability,
measured by a standard \textbf{Frailty Index (FI)}.

The following sections include:

\begin{itemize}
\tightlist
\item
  Interpretation of correlation results
\item
  Distributions of both scores
\item
  Quartile analysis
\item
  Cohen's d effect size
\item
  Figures embedded below for easy reading
\end{itemize}

As a caveat, several Longevity Score v1.1 components were not available
in NHANES 2015--2016 public data. Therefore, the present validation uses
a reduced Longevity Score, including only variables available in NHANES.
The following components are missing from the dataset and were excluded
from modeling:

\begin{itemize}
\tightlist
\item
  Coronary Artery Calcium (CAC) score
\item
  Bone mineral density (DEXA hip or spine)
\item
  Maximal VO₂ max
\item
  Heart-rate variability (HRV)
\item
  Grip strength
\item
  Epigenetic age acceleration (e.g., TruAge)
\item
  Small HDL particle count
\item
  2-hour OGTT glucose
\item
  ALT (missing in this cycle)
\item
  Detailed smoking exposure (pack-years)
\end{itemize}

\subsection{Because these variables represent key cardiometabolic,
functional, and biological aging domains, the NHANES-based LS is
expected to have lower discrimination than the full Quotient Health
Longevity
Score.}\label{because-these-variables-represent-key-cardiometabolic-functional-and-biological-aging-domains-the-nhanes-based-ls-is-expected-to-have-lower-discrimination-than-the-full-quotient-health-longevity-score.}

\subsection{\texorpdfstring{\textbf{2. Distribution of Frailty
Index}}{2. Distribution of Frailty Index}}\label{distribution-of-frailty-index}

The frailty index (FI) ranges from 0 to 1. In NHANES, older adults
cluster toward the lower end.

\subsubsection{\texorpdfstring{\textbf{Frailty Index
Distribution}}{Frailty Index Distribution}}\label{frailty-index-distribution}

\begin{figure}
\centering
\pandocbounded{\includegraphics[keepaspectratio]{fig_fi_hist.png}}
\caption{FI Histogram}
\end{figure}

\textbf{Interpretation:} Frailty is heavily right-skewed, as expected in
a general population sample. Most individuals have FI \textless{} 0.1.

\begin{center}\rule{0.5\linewidth}{0.5pt}\end{center}

\subsection{\texorpdfstring{\textbf{3. Distribution of Longevity
Score}}{3. Distribution of Longevity Score}}\label{distribution-of-longevity-score}

After normalization and credit-score mapping (300--850):

\subsubsection{\texorpdfstring{\textbf{Longevity Score
Distribution}}{Longevity Score Distribution}}\label{longevity-score-distribution}

\begin{figure}
\centering
\pandocbounded{\includegraphics[keepaspectratio]{fig_ls_hist.png}}
\caption{Longevity Histogram}
\end{figure}

\textbf{Interpretation:} The Longevity Score distributes smoothly with
no major artifacts, indicating stable scaling.

\begin{center}\rule{0.5\linewidth}{0.5pt}\end{center}

\subsection{\texorpdfstring{\textbf{4. Correlation Between Longevity
Score \& Frailty
Index}}{4. Correlation Between Longevity Score \& Frailty Index}}\label{correlation-between-longevity-score-frailty-index}

\subsubsection{\texorpdfstring{\textbf{Scatter Plot with Trend
Line}}{Scatter Plot with Trend Line}}\label{scatter-plot-with-trend-line}

\begin{figure}
\centering
\pandocbounded{\includegraphics[keepaspectratio]{fig_ls_fi_scatter.png}}
\caption{LS vs FI Scatter}
\end{figure}

\textbf{Key finding:}

\begin{itemize}
\tightlist
\item
  Pearson correlation: \textbf{--0.33}
\item
  Higher Longevity Scores tend to correspond to lower frailty levels.
\end{itemize}

This is consistent with expected relationships between physiological
risk scores and frailty constructs.

\begin{center}\rule{0.5\linewidth}{0.5pt}\end{center}

\subsection{\texorpdfstring{\textbf{5. Quartile
Analysis}}{5. Quartile Analysis}}\label{quartile-analysis}

We divided the sample into Longevity Score quartiles (Q1--Q4).

\subsubsection{\texorpdfstring{\textbf{Frailty Index Across Longevity
Score
Quartiles}}{Frailty Index Across Longevity Score Quartiles}}\label{frailty-index-across-longevity-score-quartiles}

\begin{figure}
\centering
\pandocbounded{\includegraphics[keepaspectratio]{fig_ls_fi_quartile.png}}
\caption{Quartile Boxplot}
\end{figure}

\textbf{Mean FI by quartile:}

{\def\LTcaptype{none} % do not increment counter
\begin{longtable}[]{@{}ll@{}}
\toprule\noalign{}
Quartile & Mean FI \\
\midrule\noalign{}
\endhead
\bottomrule\noalign{}
\endlastfoot
\textbf{Q1 (Lowest LS)} & 0.149 \\
Q2 & 0.084 \\
Q3 & 0.055 \\
\textbf{Q4 (Highest LS)} & 0.039 \\
\end{longtable}
}

\textbf{Interpretation:} Frailty burden decreases \textbf{monotonically}
with improving Longevity Score. The healthiest quartile shows
\textbf{3--4× lower frailty} than the least healthy group.

\begin{center}\rule{0.5\linewidth}{0.5pt}\end{center}

\subsection{\texorpdfstring{\textbf{6. Effect Size (Cohen's
d)}}{6. Effect Size (Cohen's d)}}\label{effect-size-cohens-d}

We dichotomized frailty using the standard cutoff (FI ≥ 0.25 =
``frail'').

\subsubsection{\texorpdfstring{\textbf{Longevity Score Differences by
Frailty
Group}}{Longevity Score Differences by Frailty Group}}\label{longevity-score-differences-by-frailty-group}

\begin{figure}
\centering
\pandocbounded{\includegraphics[keepaspectratio]{fig_ls_frail_groups.png}}
\caption{LS by Frailty Group}
\end{figure}

\begin{itemize}
\tightlist
\item
  \textbf{Cohen's d = 0.677}
\end{itemize}

\textbf{Interpretation:} Medium-to-large effect size. A randomly
selected non-frail adult has about a \textbf{70\% chance} of having a
higher Longevity Score than a frail adult.

This is a strong signal for a cross-sectional population dataset.

\begin{center}\rule{0.5\linewidth}{0.5pt}\end{center}

\subsection{\texorpdfstring{\textbf{7. Summary of
Findings}}{7. Summary of Findings}}\label{summary-of-findings}

\begin{itemize}
\tightlist
\item
  The Longevity Score \textbf{correlates moderately} with frailty (r ≈
  --0.33).
\item
  Quartile analysis shows \textbf{clean stepwise separation}.
\item
  The magnitude of difference between frail vs.~non-frail adults
  (\textbf{d = 0.677}) is substantial.
\item
  Overall, the score demonstrates promising discriminative performance,
  even in a non-clinical dataset with limited biomarkers.
\end{itemize}

\begin{center}\rule{0.5\linewidth}{0.5pt}\end{center}

\end{document}
